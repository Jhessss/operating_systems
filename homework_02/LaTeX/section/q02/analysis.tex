No roteiro é perguntado porque são utilizadas $\frac{n(n-1)}{2}$ \emph{threads} e não $n^2$. Isso se deve ao fato de que se fosse $n^2$ as \emph{threads} estariam comparando a própria posição com ela mesma, além de repetir combinações de números já realizadas por outras \emph{threads}. Assim, devemos analisar o fato de que devemos realizar as combinações de $N$ elementos sem repetição dessas combinações, que é definido pela fórmula: $\frac{n!}{p!(n-p)!}$. No nosso contexto $n$ é a quantidade de números entrados pelo usuário, e $p$ é igual a $2$, pois estamos combinando de $2$ em $2$. Assim ficamos:

\begin{align}
	\frac{n!}{2!(n-2)!} = \ &\frac{n(n-1)(n-2)!}{2(n-2)!}& \notag \\
	= \ &\frac{n(n-1)}{2}& \label{eq:q02}
\end{align}
